 % !TEX root = ThesisGchatzi.tex

\graphicspath{{Papers/SIGSpatial2017/}{Papers/SIGSpatial2018/}}

In this chapter, we focus on efficient evaluation of hybrid similarity search and similarity joins queries on large datasets of geolocated time series. 

\paragraph{Hybrid Similarity Search.} Consider a geolocated time series dataset that contains the user check-ins of a geosocial network. Each location is associated with the number of visitors across time. Given a query $q$, one possible query would be to retrieve all the geolocated time series within a given spatial range from that query, that are also similar in the time series domain by a given threshold. Similarly, in a geomarketing or mobile advertisement, one could identify nearby places with similar visiting pattern. Other examples involve time series generated by sensors installed at fixed locations, such as noise, weather or pollution sensors. In the \textit{DAIAD} project\footnote{\url{http://daiad.eu/}}, smart water meters are installed at households within a city to measure, analyze and mine water consumption patterns towards promoting more intelligent and efficient water use. Finding households in a region or close to a given location that exhibit consumption patterns similar to a given one can offer precious insights.

All such applications need to index geolocated time series to allow efficient similarity search based on both {\em spatial proximity} and {\em time series similarity}. Several approaches have been proposed for time series similarity, efficiently indexing large amounts of time series data. One well-studied family of approaches includes wavelet-based methods \cite{chan1999icde}, which rely on \emph{Discrete Wavelet Transform} \cite{graps1995cse} to reduce the dimensionality of time series and generate an index using the coefficients of the transformed sequences. Another line of work employs a \emph{Symbolic Aggregate Approximation} (SAX) representation of time series \cite{jessica2007dmkd}, introducing a series of indices, such as $i$SAX~\cite{shieh2008kdd}, $i$SAX 2.0~\cite{camerra2010icdm}, $i$SAX2+~\cite{camerra2014kais}, and ADS+~\cite{zoumpatianos2014sigmod}.

However, to the best of our knowledge, none of the existing works so far has considered the specific case of geolocated time series. All aforementioned indices aim at efficiently supporting similarity search for time series; in case that the analyzed time series are associated with a spatial attribute and issued queries involve spatial filters, these need to be treated independently. Thus, for queries employing both types of predicates, this implies evaluating each predicate separately. This can be done by first using a time series index to retrieve similar time series and then applying the spatial predicate on the results, or vice versa, by employing a spatial index to evaluate the spatial predicate and then filter the results according to their similarity with the query time series. In our case, we make use of our \btsr index, leveraging its hybrid indexing potential, allowing for more aggressive pruning in the spatial and time series domains simultaneously.

\paragraph{Hybrid Similarity Join.} Consider two such datasets containing time series of CO$_2$ emissions collected from two sensor networks $R$ and $S$ spread in different locations over a given spatial region. A hybrid similarity join query retrieves pairs of sensors (the first from $R$, the second from $S$) such that both the distance between the locations of the two sensors and the distance between the time series of their measurements do not exceed certain given thresholds. Then, an environmentalist may use the matching pairs to identify common patterns in nearby areas and get a better insight about the sources of pollution, its spread, etc. Similarly, check-ins in geosocial networks can also be modeled as geolocated time series and analyzed with hybrid similarity join queries. Results can indicate nearby venues with similar frequency patterns, which may be used for social recommendations according to time, place, activity, etc. Moreover, geolocated time series can indicate water or gas consumption in households. A utility company may identify nearby customers who have similar consumption profiles. Results may be used for customer segmentation, targeted marketing, planning future network upgrades, etc.

Note that similarity join differs substantially from the above hybrid similarity queries over such datasets. A hybrid similarity join query aims to identify {\em all pairs} between the two datasets qualifying to the criteria of {\em spatial proximity} and {\em time series similarity}. Clearly, performing a pairwise comparison among all pairs of objects in the two datasets is not an option when their size is large. Hence, {\em indexing} them is indispensable for efficient processing of such queries. Certainly, similarity search over indexed time series is a well-studied topic and several schemes have been proposed, as discussed in the previous paragraph. Likewise, efficient methods for distance joins in spatial databases also exist, usually over R-trees \cite{DBLP:conf/sigmod/BrinkhoffKS93, papadias1999processing}.

Our starting point is to employ such indices either for {\em time series-only} (with \isax) or {\em spatial-only} (using R-trees) filtering of candidate pairs during query evaluation. We also take advantage of the \btsr index, which enables {\em combined search} over both the time series and the spatial information of candidates and thus excels in pruning power. These algorithms concurrently traverse those indices and identify subtrees that may contain candidate matches. However, this {\em centralized} approach has certain limitations, as it cannot sustain examination of large datasets. Hence, we further suggest a space-driven data partitioning scheme that enables a {\em parallel and distributed} approach for hybrid similarity joins. Following the MapReduce paradigm, our method leverages any of the aforementioned indices to efficiently handle similarity join queries locally within each partition. This is then combined with an optimization that minimizes the amount of data transferred between worker nodes at query time without false misses. To the best of our knowledge, this is the first work to address hybrid similarity join queries over large datasets of geolocated time series.

The rest of this chapter is organized as follows. Section~\ref{sec:query_types} describes and formulates the proposed hybrid queries. Section~\ref{sec:hybrid_query_processing} presents our approach for centralized processing of hybrid similarity search and join queries. Section~\ref{sec:distributed}, introduces a parallel and distributed approach for similarity join over large datasets of geolocated time series. Section~\ref{sec:exp_btsr} reports our experimental results and, finally, Section~\ref{sec:concl_btsr_queries} concludes this chapter.