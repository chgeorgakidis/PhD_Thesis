% !TEX root = ThesisGchatzi.tex

\graphicspath{{Papers/SIGSpatial2017/}{Papers/SIGSpatial2018/}}

In this chapter, we propose a {\em hybrid index} for efficiently supporting similarity search on geolocated time series combining both spatial proximity and time series similarity. First, we introduce the \textit{\tsr}, an extension of the R-tree spatial index. In the \tsr, each node is augmented with additional information corresponding to the bounds of the time series contained in its subtree, in addition to the standard \emph{Minimum Bounding Rectangle} (MBR) denoting the spatial bound of its contents. Maintaining both kinds of bounds in each node allows to prune the search space simultaneously in the spatial dimension and in the time series dimension while traversing the index. Thus, the number of required node accesses is significantly reduced, since we only retrieve the contents of nodes that may actually contain objects satisfying both types of predicates.

Our proposed index, called \textit{\btsr} is an optimized variant of \textit{\tsr}, with its nodes having entries with more refined bounds by bundling together similar time series. This allows to compute and maintain tighter bounds for each individual bundle, hence increasing pruning effectiveness. To allow for a larger number of bundles in nodes at higher levels in the tree hierarchy, we exploit \emph{Piecewise Aggregate Approximation} \cite{keogh2001paa,faloutsos2000vldb} to trade off between the number of bundles and the resolution of the bounding time series for each bundle.

Our approach follows a similar rationale to that applied in {\em hybrid spatio-textual indices} \cite{chen2013vldb}. In that line of research, several variants of hybrid indices (e.g., the IR-tree \cite{cong2009vldb}) have been proposed to tackle the problem of combining spatial predicates with keyword search. Constructing a hybrid index that combines spatial and textual pruning has been shown to speedup processing of hybrid spatial-keyword queries. Motivated by this, our goal is to provide similar improvements for queries on geolocated time series data. Nevertheless, spatio-textual indices are designed specifically for keyword search and typically rely on inverted indices for the textual part, hence they are not applicable to queries that involve similarity of time series where the sequence of values is important.

The rest of this chapter is organized as follows. Section~\ref{subsec:preliminaries} provides background on distance functions that we use for spatial and time series domains. Sections~\ref{subsec:tsr_tree} and~\ref{sec:ctsrtree} present our proposed indices. Finally, Section~\ref{sec:concl_btsr} concludes the chapter.