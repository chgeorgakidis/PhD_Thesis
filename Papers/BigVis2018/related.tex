\subsection{Related Work}
\label{sec:related}

As our approach suggests visual exploration of time series and is based on indexing of such data, next we briefly survey both of these research topics.
\vspace{-5pt}

\paragraph{Visual Exploration of Time Series.}
In constrast to declarative visualization specifications suggested in \cite{wu2014vldb}, a recent tutorial \cite{mottin2017vldb} advocates the use of example-based methods in exploration of large relational, textual, and graph datasets. Such a {\em query-by-example} approach has been applied in \cite{eravci2013vldb} so as to explore relevance feedback for retrieval from time series databases. Instead of returning the top matching time series, this technique incorporates diversity into the results, which are presented to the user for feedback and refined in several rounds. 

RINSE \cite{zoumpatianos2015vldb} is a Recursive Interactive Series Explorer specifically designed for exploration of data series. Built on top of ADS+ \cite{zoumpatianos2014sigmod}, a special adaptive index structure for data series, it can progressively build parts of the index on demand at query time, concerning only those chunks of the data involved in users' queries. In terms of visualization, users can get those series qualifying to range or nearest-neighbor queries interactively drawn on screen, as well as monitor various statistics regarding the index footprint (e.g., RAM and disk usage) as it gets updated.

In contrast, ATLAS \cite{chan2008vast} is a visual analytics tool specifically geared towards interactivity when ad hoc filters, arbitrary aggregations, and trend exploration are applied against massive time series data. This client-server architecture employs a column store as its backend equipped with indexing, and preemptively caches data that may be required in queries so as to reduce latency when {\em panning}, {\em scrolling}, and {\em zooming} over time series.

Recently, the ONEX paradigm \cite{neamtu2016vldb} concerns online exploration of time series. It first constructs compact similarity groups over time series for specific lengths based on Euclidean distance, and then can efficiently support exploration of these groups with the Dynamic Time-Warping (DTW) method over their representatives of different lengths and alignments.

Besides, {\em smoothing} can be applied to streaming time series to remove noise in visualizations while preserving large-scale deviations \cite{rong2017vldb}. To highlight important phenomena without harming representation quality from oversmoothing, this approach introduces quantitative metrics involving variance of first differences and kurtosis to automatically calibrate smoothing parameters.

ForeCache \cite{battle2016sigmod} leverages two prefetching mechanisms to facilitate exploration of large geospatial, multidimentional and time series data stored in a DBMS. By predicting the user's behavior, it fetches the necessary data as the user interacts with the application.

None of the aforementioned methods and systems provides map-based visual exploration of {\em geolocated} time series, as is the goal of our work in this paper.
\vspace{-7pt}

%%%%%%%%%%%%%%%%%%%%%%%%%%%%%%%%%%%
\begin{comment}
\snote{{\bf CHECK:} The following are not related to time series.}

IncVisage \cite{rahman2017vldb} is an incremental visualization tool for obtaining trendlines and heatmaps over large datasets. Trading error against latency, it employs online sampling schemes in order to quickly provide a visualization encompassing salient features with error guarantees. Afterwards, at an extra overhead, this initial approximation can get incrementally improved and finally even become exact over the entire data.

ZQL is a visual exploration algebra and query language introduced in \cite{siddiqui2016vldb}, which is able to express visual patterns through an interactive GUI. When searching for such patterns (e.g., representatives or outliers) in large data collections, a graph-based query optimizer operates as a wrapper over any relational DBMS used as a backend repository.

FlashView \cite{pang2017vldb} is a visual data explorer that can directly query raw data files and estimate a series of aggregates in real-time. Those correlated estimates can be depicted into charts through a user interface.

Foresight \cite{demiralp2017vldb} is a platform that facilitates discovery of visual insights (e.g., dispersion, skewness, outliers, linear correlation) from large high-dimensional datasets. Instead of directly exploring data dimensions and visual encodings, this approach handles those insights as first-class citizens in its model and offers query tools and metrics for their effective discovery after a necessary preprocessing step that provides approximate descriptors of the data (sketches, samples).

\end{comment}
%%%%%%%%%%%%%%%%%%%%%%%%%%%%%%%%%%

\paragraph{Indexing of Time Series.} Earlier approaches towards indexing of time series data were based on leveraging multi-resolution representations. For instance, the Discrete Wavelet Transform \cite{graps1995cse} is used in \cite{chan1999icde} to gradually reduce the dimensionality of time series data via the \emph{Haar wavelet} \cite{haar1910theorie} and generate an index using the coefficients of the transformed sequences. In \cite{popivanov2002icde}, it is further observed that, other than orthonormal wavelets, bi-orthonormal ones can also be used for efficient similarity search over wavelet-indexed time series data, demonstrating several such wavelets that outperform the Haar wavelet in terms of precision and performance. In addition, an alternative approach regarding $k$-nearest neighbor search over time series data is introduced in \cite{kashyap2011kdd}. The proposed method accesses the coefficients of Haar-wavelet-transformed time series through a sequential scan over step-wise increasing resolutions.

State-of-the-art approaches for time series indexing comprise methods based on the {\em Symbolic Aggregate Approximation} (SAX) representation \cite{jessica2007dmkd}. This is derived from the {\em Piecewise Aggregate Approximation} (PAA) representation of a time series \cite{keogh2001paa,faloutsos2000vldb}, by quantizing the segments of its PAA representation on the $y$-axis. The first attempt to leverage the potential of the SAX representation was presented in \cite{shieh2008kdd}, introducing the indexable Symbolic Aggregate Approximation ($i$SAX), capable of a multi-resolution representation for time series. The iSAX index was further extended to $i$SAX 2.0  \cite{camerra2010icdm} by enabling bulk loading of time series data. Its next version is the $i$SAX2+ index \cite{camerra2014kais}, which handles better the expensive I/O operations caused by the aggressive node splitting while building the index. Finally, the ADS+ index \cite{zoumpatianos2014sigmod} is another extension of $i$SAX, which overcomes the still significantly expensive index build time by adaptively building the index while processing the workload of queries issued by the user. A comprehensive overview of the time series indexing approaches based on the SAX representation is presented in \cite{palpanas2016bigsm}.

Unfortunately, none of the abovementioned access methods can inherently support geolocated time series, i.e., time series inextricably associated with a location. To the best of our knowledge, the only index in the literature that supports such time series is the \btsr index \cite{chatzig17btsr}. This hybrid index follows a similar rationale set by {\em spatio-textual indices} \cite{chen2013pvldb,cong2009vldb,defelipe2008icde,chen2006sigmod} that have been proposed to speed up evaluation of queries combining location-based predicates with keyword search. Essentially, this paradigm implies combining a spatial index structure (e.g., R-tree, Quadtree, Space-Filling Curve) with a textual index (e.g., inverted file, signature file). Depending on their structure, these variants can be characterized either as {\em spatial-first} or {\em textual-first} indices \cite{christoforaki2011cikm}.
In a similar spirit, our \btsr is a spatial-first index based on the R-tree that can additionally abstract similarity of time series instead of a textual one. As a result, it can offer analogous improvements when searching against geolocated time series data, as we discuss in more detail in Section~\ref{subsec:btsr}.

\begin{comment}

\paragraph{Spatio-Textual Indices.} There is an increasing amount of spatio-textual objects, e.g., Points of Interest (PoI) with textual descriptions, geotagged tweets or posts in social media, etc. This has motivated research on hybrid spatial-keyword queries combining location-based predicates with keyword search. Main query types include the \emph{Boolean Range Query}, which retrieves all objects that contain a given set of keywords and are located within a specified spatial range; the \emph{Boolean $k$NN Query}, which returns the $k$ nearest objects to a specific location and contain the given keywords; and the \emph{Top-$k$ $k$NN Query}, which finds the top-$k$ objects according to an objective function that assigns hybrid scores to objects based on both their keyword similarity and spatial proximity to the query object \cite{chen2013pvldb}.

To evaluate such queries efficiently, the main idea is to construct hybrid index structures that simultaneously partition the data in both dimensions, spatial and textual. Essentially, this implies combining a spatial index structure (e.g., R-tree, Quadtree, Space-Filling Curve) with a textual index (e.g., inverted file, signature file). Depending on their form, the resulting variants can be characterized either as {\em spatial-first} or {\em textual-first} indices \cite{christoforaki2011cikm}. One of the most fundamental and characteristic ones is the IR-tree \cite{cong2009vldb,zhisheng2011tkde}, which extends the R-tree by augmenting the contents of each node with a pointer to an inverted file indexing terms and documents contained in its sub-tree. Several other hybrid spatio-textual indices extending the R-tree (or R$^*$-tree) have been proposed, such as the IR$^2$-tree \cite{defelipe2008icde}, the KR$^*$-Tree \cite{hariharan2007ssdbm}, SKI \cite{cary2010ssdbm} and S2I \cite{rocha2011ssd}, while methods based on space filling curves include SF2I \cite{chen2006sigmod} and SFC-QUAD \cite{christoforaki2011cikm}.



\paragraph{Geolocated Time Series Indexing.} 

\btsr is based on the R-tree \cite{Guttman1984} for the spatial indexing part. Recall that an R-tree organizes a hierarchy of nested $d$-dimensional rectangles. Each node corresponds to a disk page and represents the MBR of its children or, for leaf nodes, the MBR of its contained geometries. The number of entries per node (excluding the root) is between a lower bound $m$ and a maximum capacity $M$. Query execution in R-trees starts from the root. MBRs in any visited node are tested for intersection against a search region. Qualifying entries are recursively visited until the leaf level or until no further overlaps are found. Several paths may be probed, because multiple sibling entries could overlap with the search region. \btsr extends the information that is stored within the nodes of the R-Tree with bundles of MBTSs, in order to allow for efficient pruning during hybrid queries hybrid queries by combining time series similarity with spatial proximity. To efficiently support the $Q_{rb}$ query, we have extended the \btsr, by also storing within each node the number of geolocated time series that reside in its sub-tree.

\end{comment}
