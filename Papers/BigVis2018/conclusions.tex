\subsection{Conclusions and Future Work}
\label{sec:conclusions}

In this paper, we introduced a method for map-based visual exploration of geolocated time series data. To that end, we proposed a summarization approach over geolocated time series, which allows a visual analytics application to retrieve the required information. Such retrieval can be achieved at low latency, thus being suitable for interactive exploration of large volumes of such data. The results can be displayed on a map, depicting the relevant MBRs and the number of time series contained in each one, for a selected pattern detected in the time series data. Thanks to the support of a robust hybrid indexing technique, the patterns detected at a given zoom level are calculated via $k$-means clustering over the time series that reside in the currently visible part of the map. Our experiments on a large-scale synthetic dataset indicated that the visualization can be rendered adequately fast for use in interactive map-based applications. Additionally, we presented indicative demonstrations of the visualizations generated on two real-world datasets from different domains, confirming that these visualizations are helpful in revealing patterns both on the time series themselves as well as their geographic distribution. 

Our ongoing and future work focuses on supporting more detailed visual analytics and identifying more fine-grained patterns through visual exploration. One possible extension would be to enable zooms along time, so that the user can identify patterns and their spatial distribution, not only over the entire time series, but also over particular intervals. Further, it would be interesting to drill-down in a particular summarized result and discover whether there are differentiations in the spatial distributions of its constituent, more detailed patterns.