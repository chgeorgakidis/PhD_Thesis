\subsubsection{Related Work}
\label{sec:related}

%As our approach suggests visual exploration of time series and is based on indexing of such data, next we briefly survey both of these research topics.

In the following, we review existing approaches regarding indexing and visual exploration of time series.

\paragraph{Indexing of Time Series} Earlier approaches towards indexing of time series data were often based on leveraging multi-resolution representations. For instance, the Discrete Wavelet Transform \cite{graps1995cse} is used in \cite{chan1999icde} to gradually reduce the dimensionality of time series data via the Haar wavelet and generate an index using the coefficients of the transformed sequences. In \cite{popivanov2002icde}, it is further observed that, other than orthonormal wavelets, bi-orthonormal ones can also be used for efficient similarity search over wavelet-indexed time series data, demonstrating several such wavelets that outperform the Haar wavelet in terms of precision and performance. In addition, an alternative approach regarding $k$-nearest neighbor search over time series data is introduced in \cite{kashyap2011kdd}. The proposed method accesses the coefficients of Haar-wavelet-transformed time series through a sequential scan over step-wise increasing resolutions.

State-of-the-art approaches for time series indexing comprise methods based on the {\em Symbolic Aggregate Approximation} (SAX) representation \cite{jessica2007dmkd}. This is derived from the {\em Piecewise Aggregate Approximation} (PAA) representation of a time series \cite{keogh2001paa,faloutsos2000vldb}, by quantizing the segments of its PAA representation on the $y$-axis. The first attempt to leverage the potential of the SAX representation was presented in \cite{shieh2008kdd}, introducing the indexable Symbolic Aggregate Approximation (\isax), capable of a multi-resolution representation for time series. The \isax index was further extended to \isax 2.0  \cite{camerra2010icdm} by enabling bulk loading of time series data. Its next version is the \isax2+ index \cite{camerra2014kais}, which handles better the expensive I/O operations caused by the aggressive node splitting while building the index. Finally, the ADS+ index \cite{zoumpatianos2014sigmod} is another extension of \isax, which overcomes the still significantly expensive index build time by adaptively building the index while processing the workload of queries issued by the user. A comprehensive overview of time series indexing approaches based on SAX representation is presented in \cite{palpanas2016bigsm}.

Unfortunately, none of the abovementioned access methods can inherently support geolocated time series, i.e., time series inextricably associated with a location. To the best of our knowledge, the only index in the literature that supports such time series is the \btsr index~\cite{chatzig17btsr}. This hybrid index follows a similar rationale set by {\em spatio-textual indices} \cite{chen2013pvldb} that have been proposed to speed up evaluation of queries combining location-based predicates with keyword search. Essentially, this paradigm implies combining a spatial index structure (e.g., R-tree, Quadtree, Space-Filling Curve) with a textual index (e.g., inverted file, signature file). Depending on their structure, these variants can be characterized either as {\em spatial-first} or {\em textual-first} indices \cite{christoforaki2011cikm}. In a similar spirit, our \btsr is a spatial-first index based on the R-tree that can additionally abstract similarity of time series instead of a textual one. As a result, it can offer analogous improvements when searching against geolocated time series data, as we discuss in more detail in Section~\ref{subsec:btsr}.

\paragraph{Visual Exploration of Time Series} Numerous approaches attempt to leverage the potential of summarizing or aggregating the information of large time series data to facilitate visual exploration and knowledge extraction. An early approach is~\cite{mintz1997tracking}, where the authors use tile maps and box plots to discover ten-year trends in air pollution data. In \cite{keim1995recursive}, the authors introduce a pixel-oriented visualization to detect recursive patterns, where each data value is represented by one pixel. The authors demonstrate the potential of their method using a stock market dataset. An extension of this work is presented in~\cite{lammarsch2009hierarchical}, where several time granularities are combined in a single visualization to enhance the knowledge extraction potential of recursive patterns. 

Of particular interest are visualization approaches that attempt to leverage the potential of declarative SQL-like languages and DBMSs to enable exploratory queries. Such an approach is suggested in M4~\cite{jugel2014m4}, where the authors introduce an aggregation-based dimensionality reduction scheme for visualizing horizontally large time series using line charts. Their approach operates on top of an RDBMS and supports various SQL queries that select and visualize particular parts of time series. ForeCache \cite{battle2016sigmod} leverages two prefetching mechanisms to facilitate exploration of large geospatial, multidimensional and time series data stored in a DBMS. By predicting the user's behavior, it fetches the necessary data as the user interacts with the application. Another declarative language-based visualization is suggested in \cite{wu2014vldb}, where relational algebra queries are used to represent the visualization, leveraging the potential of traditional and visualization-specific optimizations. In contrast, a recent tutorial \cite{mottin2017vldb} advocates the use of example-based methods in exploration of large relational, textual, and graph datasets. Such a {\em query-by-example} approach has been applied in \cite{eravci2013vldb} so as to explore relevance feedback for retrieval from time series databases. Instead of returning the top matching time series, this technique incorporates diversity into the results, which are presented to the user for feedback and refined in several rounds.

RINSE \cite{zoumpatianos2015vldb} is a Recursive Interactive Series Explorer specifically designed for exploration of data series. Built on top of ADS+ \cite{zoumpatianos2014sigmod}, a special adaptive index structure for data series, it can progressively build parts of the index on demand at query time, concerning only those chunks of the data involved in users' queries. In terms of visualization, users can get those series qualifying to range or nearest-neighbor queries interactively drawn on screen, as well as monitor various statistics regarding the index footprint (e.g., RAM and disk usage) as it gets updated. In contrast, ATLAS \cite{chan2008vast} is a visual analytics tool specifically geared towards interactivity when ad hoc filters, arbitrary aggregations, and trend exploration are applied against massive time series data. This client-server architecture employs a column store as its backend equipped with indexing, and preemptively caches data that may be required in queries so as to reduce latency when {\em panning}, {\em scrolling}, and {\em zooming} over time series. Recently, the ONEX paradigm \cite{neamtu2016vldb} concerns online exploration of time series. It first constructs compact similarity groups over time series for specific lengths based on Euclidean distance, and then can efficiently support exploration of these groups with the Dynamic Time-Warping (DTW) method over their representatives of different lengths and alignments. {\em Smoothing} can be applied to streaming time series to remove noise in visualizations while preserving large-scale deviations \cite{rong2017vldb}. To highlight important phenomena without harming representation quality from oversmoothing, this approach introduces quantitative metrics involving variance of first differences and kurtosis to automatically calibrate smoothing parameters.

The ability to zoom in to specific parts of interest of a large time series can significantly enhance the exploratory potential of a visualization. Stack zooming~\cite{javed2010stack} provides such a functionality, by building hierarchies of line chart visualizations for user-defined intervals on large time series data. Each selected interval is zoomed and stacked beneath the initial time series. A similar approach is KronoMiner~\cite{zhao2011kronominer}, which employs a radial-based visualization to enable zooming functionality for specific time intervals. The interface is visually refined through an iterative design procedure involving expert user feedback. ChronoLenses \cite{zhao2011exploratory} introduces a domain-independent visualization that offers the ability to perform on-the-fly transformations (e.g., Fourier transform, auto-correlation) of the selected interval using {\em lenses}.

Zooming in regions of interest in time series can be performed via {\em timeboxes}, which essentially consist of rectangular regions on the time series domain thus specifying intervals in both the time and value axis. The procedure retrieves the time series whose values in the given region are fully contained in the rectangle. Hochheister et al. introduced timeboxes~\cite{hochheiser2004dynamic} along with {\em TimeSearcher}, an application for visual exploration of time series datasets that implements timebox queries. The user is able to draw rectangles on the time series domain and the results are separately displayed on-screen. Keogh et al.~\cite{keogh2002augmented} extended the timeboxes, introducing the {\em Variable Time Timeboxes}, which allowed a degree of uncertainty in the time axis. Later versions of TimeSearcher (such as ~\cite{aris2005representing}) provided enhanced functionality, allowing the visual exploration of longer time series ($>$10,000 time points) and offering forecasting functionality.

Contrary to our approach, none of the aforementioned methods and systems provides map-based visual exploration of {\em geolocated} time series. In our recent work~\cite{chatzigeorgakidis2018map}, we have introduced a summary construction method for geolocated time series, that utilizes our spatial-first \btsr, to enable spatial-domain map-based exploratory visualizations. In this paper, we augment this work by introducing a {\em time series-first} hybrid index to facilitate timebox search on both horizontally and vertically large time series datasets. We enable efficient exploration on geolocated time series datasets, by timely executing user-defined timebox search, enabling the exploration also in the time series domain.