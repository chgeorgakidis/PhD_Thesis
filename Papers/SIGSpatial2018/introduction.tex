%Motivation
%Time series data is a treasure trove for a variety of mining and monitoring applications both in industry (e.g., finance, public utilities) and in academia (e.g., astronomy, biology), while a rapidly increasing bulk of such data is also generated on the Web and the Internet of Things. Although indexing, analysis and exploration of time series data has attracted a lot of interest from the database and data mining communities \cite{camerra2014kais,ding2008pvldb,shieh2008kdd}, studying of {\em geolocated time series} only lately has come under focus \cite{chatzig17btsr}. This refers to time series that are produced at, or associated with, a specific geolocation. Analyzing such geolocated time series can offer insights regarding trends and patterns. For example, such data can be used to represent water consumption measured by smart meters installed in households. In fact, they abound in many applications and are often used to identify user check-ins patterns in geosocial networks, weather or pollution measurements from a sensor network over a geographical area, fluctuations of house prices in real estate, and so on.

Time series data is a treasure trove for a variety of mining and monitoring applications both in industry (e.g., finance, public utilities) and in academia (e.g., astronomy, biology), while a rapidly increasing bulk of such data is also generated on the Web and the Internet of Things. Although indexing, analysis and exploration of time series data has attracted a lot of interest from the database and data mining communities \cite{camerra2014kais,ding2008pvldb,shieh2008kdd}, studying of {\em geolocated time series} only lately has come under focus \cite{chatzig17btsr}. This refers to time series that are produced at, or associated with, a specific geolocation. Analyzing such data can offer insights regarding trends and patterns in many applications. Indeed, they are often used to identify user check-in patterns in geosocial networks, weather or pollution measurements from a sensor network, resource consumption in households, fluctuations of house prices in real estate, and so on.

%Query specification
%In this work, we focus on efficient evaluation of {\em hybrid similarity join queries} between large datasets of geolocated time series. Consider two such datasets containing time series of CO$_2$ emissions collected from two sensor networks $R$ and $S$ spread in different locations over the same geographical region. A hybrid similarity join query retrieves pairs of sensors (the first from $R$, the second from $S$) such that: (i) the distance between the locations of the two sensors and (ii) the distance between the time series of their measurements do not exceed certain given thresholds. Then, an environmentalist may use the matching pairs to identify common patterns in nearby areas and get a better insight about the sources of pollution, its spread, etc. Similarly, check-ins in geosocial networks can also be modeled as geolocated time series and analyzed with hybrid similarity join queries. Results can indicate nearby venues with similar frequency patterns, so they may be used for social recommendations according to time, place, activity, etc. Moreover, geolocated time series can indicate water or gas consumption in households. A utility company may identify customers in nearby locations who have similar consumption profiles. Results may be used for customer segmentation and targeted marketing, for planning future network upgrades, etc. The latter example involves a single dataset, i.e., it is a hybrid similarity {\em self-join} query.

%Query specification
In this work, we focus on efficient evaluation of {\em hybrid similarity join queries} between large datasets of geolocated time series. Consider two such datasets containing time series of CO$_2$ emissions collected from two sensor networks $R$ and $S$ spread in different locations over a given spatial region. A hybrid similarity join query retrieves pairs of sensors (the first from $R$, the second from $S$) such that both the distance between the locations of the two sensors and the distance between the time series of their measurements do not exceed certain given thresholds. Then, an environmentalist may use the matching pairs to identify common patterns in nearby areas and get a better insight about the sources of pollution, its spread, etc. Similarly, check-ins in geosocial networks can also be modeled as geolocated time series and analyzed with hybrid similarity join queries. Results can indicate nearby venues with similar frequency patterns, which may be used for social recommendations according to time, place, activity, etc. Moreover, geolocated time series can indicate water or gas consumption in households. A utility company may identify nearby customers who have similar consumption profiles. Results may be used for customer segmentation, targeted marketing, planning future network upgrades, etc.

%Challenges
%Note that this type of query differs substantially from the hybrid queries over such datasets addressed in \cite{chatzig17btsr}. In that case, given a geolocated time series $q$, the goal is to identify those similar to $q$ in both the spatial and the time series domains. Instead, a hybrid similarity join query aims to identify {\em all pairs} between the two datasets qualifying to the criteria of {\em spatial proximity} and {\em time series similarity}. Clearly, performing a pairwise comparison among all pairs of objects in the two datasets is not an option when their size is large. Hence, {\em indexing} them is indispensable for efficient processing of such queries. Certainly, similarity search over indexed time series is a well-studied topic and several schemes have been proposed, like wavelet-based methods~\cite{chan1999icde} or the family of \isax trees~\cite{shieh2008kdd,camerra2010icdm,camerra2014kais,zoumpatianos2014sigmod}. Likewise, efficient methods for distance joins in spatial databases also exist, usually over R-trees \cite{DBLP:conf/sigmod/BrinkhoffKS93, papadias1999processing}.

%Challenges
A hybrid similarity join query aims to identify {\em all pairs} between the two datasets qualifying to the criteria of {\em spatial proximity} and {\em time series similarity}. Clearly, performing a pairwise comparison among all pairs of objects in the two datasets is not an option when their size is large. Hence, {\em indexing} them is indispensable for efficient processing of such queries. Certainly, similarity search over indexed time series is a well-studied topic and several schemes have been proposed, like wavelet-based methods~\cite{chan1999icde} or the family of \isax trees~\cite{shieh2008kdd,camerra2010icdm,camerra2014kais,zoumpatianos2014sigmod}. Likewise, efficient methods for distance joins in spatial databases also exist, usually over R-trees \cite{DBLP:conf/sigmod/BrinkhoffKS93, papadias1999processing}. 

%General approach
In this paper, our starting point is to employ such indices either for {\em time series-only} (with \isax) or {\em spatial-only} (using R-trees) filtering of candidate pairs during query evaluation. We also take advantage of the \btsr index \cite{chatzig17btsr}, which enables {\em combined search} over both the time series and the spatial information of candidates and thus excels in pruning power. These algorithms concurrently traverse those indices and identify subtrees that may contain candidate matches. However, this {\em centralized} approach has certain limitations, as it cannot sustain examination of large datasets. Hence, we further suggest a space-driven data partitioning scheme that enables a {\em parallel and distributed} approach for hybrid similarity joins. Following the MapReduce paradigm, our method leverages any of the aforementioned indices to efficiently handle similarity join queries locally within each partition. This is then combined with an optimization that minimizes the amount of data transferred between worker nodes at query time without false misses.

%Contribution
To the best of our knowledge, this is the first work to address hybrid similarity join queries over large datasets of geolocated time series. Our main contributions can be summarized as follows:

\begin{itemize}
%\item We introduce the novel problem of hybrid similarity joins between datasets of geolocated time series.
 \item We adapt state-of-the-art indices over such data in centralized settings and propose traversal methods that can prune the search space and return answers without false misses.
 \item We suggest a space-driven partitioning method to distribute large datasets in cluster infrastructures, thus enabling faster, in-parallel evaluation of smaller similarity join tasks.
 \item We conduct an extensive experimental evaluation against large data volumes of geolocated time series, confirming that our methods can efficiently and correctly process hybrid similarity join queries in these settings.
\end{itemize}

%Roadmap
The rest of the paper is organized as follows. Section \ref{sec:related} reviews related work. Section \ref{sec:problem} describes the problem. Section \ref{sec:preliminaries} provides background on two state-of-the-art index methods that we employ in alternative similarity join strategies presented in Section \ref{sec:centralized}. Section \ref{sec:distributed} introduces a parallel and distributed approach for similarity join over large datasets of geolocated time series. Section \ref{sec:evaluation} reports our experimental results, and Section \ref{sec:conclusions} concludes the paper.