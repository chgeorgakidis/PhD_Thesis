\subsection{Related Work}
\label{sec:related}

Earlier approaches on {\em time series indexing} have leveraged multi-resolution representations to gradually reduce time series dimensionality with Discrete Wavelet Transform and then index the resulting coefficients \cite{chan1999icde, popivanov2002icde}. Current state-of-the-art indexing over time series involves the {\em indexable Symbolic Aggregate Approximation} (\isax) family of trees, which are based on the {\em Symbolic Aggregate Approximation} (SAX) representation of each time series \cite{jessica2007dmkd}. After the original \isax tree introduced in \cite{shieh2008kdd}, several extensions have been proposed, including \isax 2.0 \cite{camerra2010icdm} and $i$SAX2+ \cite{camerra2014kais}, which enable bulk loading of time series data and better handle the expensive I/O operations caused by aggressive node splitting during index construction. The ADS+ index \cite{zoumpatianos2014sigmod} is an adaptive approach of \isax, built progressively while processing query workloads, thus sparing much of the initial construction overhead. A comprehensive overview over time series indexing schemes based on the SAX representation is available in \cite{palpanas2016bigsm}. However, efficiently accommodating spatial information in any such scheme is not straightforward.

%In \cite{kashyap2011kdd}, Kashyap at al. utilize multi-resolution representations to introduce an alternative approach to the $k$-nearest neighbor search over time series data by accessing the coefficients of Haar-wavelet-transformed time series through a sequential scan over step-wise increasing resolutions.


%\noindent \emph{Time Series Indexing.} Earlier approaches towards indexing time series data were based on leveraging multi-resolution representations. For instance, the Discrete Wavelet Transform \cite{graps1995cse} is used in \cite{chan1999icde} to gradually reduce the dimensionality of time series data via the \emph{Haar wavelet} \cite{haar1910theorie} and generate an index using the coefficients of the transformed sequences. In \cite{popivanov2002icde}, it is further observed that, other than orthonormal wavelets, bi-orthonormal ones can also be used for efficient similarity search over wavelet-indexed time series data, demonstrating several such wavelets that outperform the Haar wavelet in terms of precision and performance. In addition, an alternative approach to the $k$-nearest neighbor search over time series data is introduced in \cite{kashyap2011kdd}. The proposed method accesses the coefficients of Haar-wavelet-transformed time series through a sequential scan over step-wise increasing resolutions.

%State-of-the-art approaches for time series indexing comprise methods based on the {\em Symbolic Aggregate Approximation} (SAX) representation \cite{jessica2007dmkd}. This is derived from the {\em Piecewise Aggregate Approximation} (PAA) representation of a time series \cite{keogh2001paa,faloutsos2000vldb}, by quantizing the segments of its PAA representation on the $y$-axis. The first attempt to leverage the potential of the SAX representation was presented in \cite{shieh2008kdd}, introducing the indexable Symbolic Aggregate Approximation ($i$SAX), capable of a multi-resolution representation for time series. The iSAX index was further extended to $i$SAX 2.0 in \cite{camerra2010icdm} by enabling bulk loading of time series data. Its next version is the $i$SAX2+ index \cite{camerra2014kais}, which handles better the expensive I/O operations caused by the aggressive node splitting while building the index. Finally, the ADS+ index \cite{zoumpatianos2014sigmod} is another extension of $i$SAX, which attempts to overcome the still significantly expensive index build time by adaptively building the index while processing the workload of queries issued by the user. A comprehensive overview and comparison of the time series indexing approaches based on the SAX representation is presented in \cite{palpanas2016bigsm}.

%However, none of the above approaches supports geolocated time series, and thus cannot efficiently process hybrid queries combining time series similarity with spatial proximity.

With respect to {\em spatial join queries}, several methods have been proposed, often based on the R-tree family of indices \cite{Guttman1984, Beckmann1990}. In particular, the spatial join algorithms over R$^*$-trees introduced in \cite{DBLP:conf/sigmod/BrinkhoffKS93} can minimize the CPU and I/O cost in searching. {\em Multiway} spatial joins \cite{papadias1999processing} generalize search over more than two R-trees. Top-$k$ spatial distance joins~\cite{qi2013efficient} employ R-tree-based spatial joins in data blocks ordered by an objective score to retrieve $k$ pairs of objects with highest score. However, all such algorithms are applied against spatial information only. Based on a similar observation for answering a variety of queries over geolocated time series, in \cite{chatzig17btsr} we proposed the \btsr, a hybrid index based on the R-tree, but having nodes that also store bounds over the time series information in their underlying subtree. This index offers increased pruning capabilities for queries involving both time series similarity and spatial proximity. However, handling hybrid similarity joins is not addressed in \cite{chatzig17btsr}; we develop such a method next in this paper.

Our current work on geolocated time series data is reminiscent of related approaches in {\em spatio-textual search}. Spatio-textual joins identify objects that are both spatially and textually close. In particular, the algorithm proposed in \cite{Bouros:2012:SSJ:2428536.2428537} uses a spatial partitioning in conjunction with spatial joins over R-trees in order to batch process such queries. MapReduce-based methods in \cite{Zhang:2014:ESS:2682647.2682773} resolve spatio-textual joins on spatially partitioned data. However, it should be stressed that time series information is quite distinct from documents or keywords used in those works and certainly requires a totally different processing paradigm. To the best of our knowledge, ours is the first approach for processing similarity joins on geolocated time series data.