% the water problem
Water is one of the most valuable resources for mankind. Experts estimate that water will be the most useful good in the decades to come, and combined with climate change, a potential cause for geopolitical tensions and conflicts. Drinkable water is only a small percentage of existing water bodies and the significance of water management is greatly acknowledged by various local and national policies. To determine how water demand is formulated, identify the factors that influence it, and forecast future demand, the availability, collation, and analysis of water consumption data is considerably required.

% the lack of data and analysis services
In the energy domain, huge amounts of data are generated and feed many applications regarding billing, grid and demand management in a fine-grained or a more coarse-grained fashion. However, the available data for water consumption still exhibit extremely low temporal and spatial granularity. Also, they are highly aggregated and complex, limiting the potential for data mining and analysis services.

% the Big picture
Given the current low volume of data derived from water measurements, existing systems cannot scale to manage water consumption data. Towards this direction, \textit{smart water meters} are installed either on single fixtures or on a residence supply system and provide us with a wealth of such data. But still, consumers have limited access to analytics and information concerning their usage patterns and habits. Furthermore, personal water monitoring and typical metering infrastructures must be decoupled, but also interoperable. This way, they can provide a better understanding of water use and influence sustainable consumer behaviour.

During the last few years, distributed computing technologies have emerged to satisfy the need for systems that can efficiently manage massive volumes of data. \textit{MapReduce} is a programming model that enables execution of algorithms on a cluster based environment, through mapping elements of a dataset in a key-value pair perception (i.e.~\textit{mappers}) to several machines (i.e.~\textit{reducers}), via a hashing function. The result is stored in a \textit{Distributed File System} (DFS). Several open-source frameworks of the MapReduce programming model exist, including \textit{Hadoop}\footnote{https://hadoop.apache.org/}, \textit{Spark}\footnote{https://spark.apache.org/} and \textit{Flink}\footnote{https://flink.apache.org/}. Flink can operate over Hadoop DFS (HDFS) and exhibits better performance on MapReduce algorithms compared to the other frameworks~\cite{studybig2014}.

Forecasting tasks are often performed by applying machine learning algorithms over large data collections. The simplicity and the effectiveness of \textit{$k$-Nearest Neighbours} ($k$-NN), have fed many research communities with numerous applications and scientific approaches, which evolve its potential over the years. Of particular interest are the $k$-NN join methods~\cite{bohm2004knn}, which retrieve the nearest neighbours of \textit{every} element in a testing dataset ($R$) from a set of labelled elements in a training dataset ($S$). Each data element consists of several \textit{features}, which constitute the preliminary knowledge on which the classification is conducted. However, computing $k$-NN joins on huge amounts of data is time consuming when conducted by a single CPU, as it requires $k$-NN computation for every element in dataset $R$. The MapReduce model accelerates this computation, allowing for parallel and distributed execution. 

This work addresses the challenges of Big Data management, analysis and interpretation of massive water consumption data collected by smart meters. It proposes a method for scalable data analysis and knowledge extraction of diverse Big Data collections. It also devises novel means which facilitate consumers and water utilities to draw useful conclusions, lead them towards efficient water usage and induce sustainable lifestyles. We focus on water consumption classification and forecasting, by using large volumes of historical water consumption data.

In this paper, we present the Flink zkNN (\textit{F-zkNN} for short), a robust probabilistic classifier for parallel and distributed execution by extending over the \textit{H-zkNN}~\cite{zhang2012epk}. More particularly, we propose a new $k$-NN based probabilistic classifier which identifies and predicts water consumption class probability for arbitrary temporal basis, by using a voting scheme. We introduce a MapReduce based approach which reduces file operations for large amounts of data and is uniquely initialized upon launch. Our approach is unified in a single session to reduce space occupation and cluster overloading. Through an experimental evaluation we show that the proposed method efficiently achieves high prediction precision and useful knowledge extraction.