\subsection{Related Work}
\label{sec:related}

%Time series similarity and indexing

Similarity search over time series has provided a wide range of algorithmic approaches; a detailed survey with experimental evaluation is available in~\cite{DBLP:journals/pvldb/EchihabiZPB18}. Initially, the focus was mostly on wavelet-based methods~\cite{chan1999icde} to reduce the dimensionality of time series and generate an index based on the transformed sequences. In contrast, state-of-the-art approaches for time series indexing are based on the {\em Symbolic Aggregate Approximation} (SAX) representation \cite{jessica2007dmkd}. The first index in this family was $i$SAX\cite{shieh2008kdd}, offering multi-resolution representations for time series. Further extensions like $i$SAX 2.0~\cite{camerra2010icdm}, $i$SAX2+~\cite{camerra2014kais}, ADS+~\cite{zoumpatianos2014sigmod}, Coconut~\cite{DBLP:journals/pvldb/KondylakisDZP18}, DPiSAX~\cite{dpisaxjournal}, and ParIS~\cite{DBLP:conf/bigdataconf/PengFP18} provided a wide range of advanced capabilities. These indices support {\em global} similarity search, i.e., the similarity score is computed over the entire length of the compared time series, as opposed to {\em local} similarity, which allows to consider similar subsequences. The most recent addition to this $SAX$-based family is \textit{ULISSE}~\cite{linardi2018scalable}, which can answer similarity search queries of {\em varying} length. However, this still differs from our setting, since in \textit{ULISSE} the goal is to build an index that supports similarity search for queries of any length within a given range $[\ell_{min}, \ell_{max}]$. Furthermore, none of the aforementioned approaches supports geolocated time series, and thus cannot efficiently process hybrid queries combining conditions on spatial distance and time series similarity.
% \checknote{The most recent addition to this $SAX$-based family is \textit{ULISSE}~\cite{linardi2018scalable}, which can answer similarity search queries of {\em varying} length, which can be less than the entire length of the given time series. However, this still differs from our setting, since in \textit{ULISSE} the query spans the full duration of the underlying data and local similarities may be identified anywhere therein.}


%Subsequence matching

The problem of {\em subsequence matching} over time series is to identify matches of a (relatively short) query subsequence across one or more (relatively long) time series. The UCR suite \cite{rakthanmanon2012searching} offers a framework comprising various optimizations regarding subsequence similarity search. Matrix Profile~\cite{yeh2016matrix} includes methods for detecting, for each subsequence of a time series, its \textit{nearest neighbor} subsequence, by keeping track of Euclidean distances among candidate pairs. Applying such approaches in our setting is not straightforward. First, they involve Euclidean or DTW distances, which are different from our definition of local similarity score, hence the pruning heuristics do not hold in our case. Second, they do not consider geolocated time series, thus spatial filtering has to be carried out independently, which reduces pruning opportunities.



%Hybrid indexing with \btsr

To the best of our knowledge, the only index that supports searching over geolocated time series is the \btsr~\cite{chatzig17btsr,DBLP:conf/gis/Chatzigeorgakidis18}. This hybrid index follows a similar rationale set by {\em spatio-textual indices} \cite{chen2013pvldb} that can facilitate evaluation of queries combining location-based predicates with keyword search. 
%\checknote{{\bf Remove for space?} That paradigm combines a spatial index structure (e.g., R-tree, Quadtree, Space-Filling Curve) with a textual index (e.g., inverted file, signature file).} 
In a similar spirit, \btsr is a spatial-first index based on the R-tree that can additionally compute bounds on similarity of time series instead of a textual similarity between documents. Apart from an MBR, each node also stores bounds over the time series indexed in its subtree. Thus, it offers increased pruning capabilities for range and top-$k$ queries involving both time series similarity and spatial proximity. In the current work, we show how \btsr can be used for another family of hybrid queries involving {\em local similarity} of time series. Furthermore, we introduce a variant structure, called \sbtsr, which constructs tighter bounds over temporally segmented time series to offer stronger pruning power.
% Moreover, it efficiently supports hybrid similarity joins over geolocated time series both in centralized and distributed settings \cite{DBLP:conf/gis/Chatzigeorgakidis18}. 




%%%%%%%%%%%%%%%%%%%%%%%%%%%%%%%%%%

\eat{
\checknote{\noindent \emph{Time Series Indexing.} Earlier approaches towards indexing time series data were based on leveraging multi-resolution representations. For instance, the Discrete Wavelet Transform \cite{graps1995cse} is used in \cite{chan1999icde} to gradually reduce the dimensionality of time series data via the \emph{Haar wavelet} \cite{haar1910theorie} and generate an index using the coefficients of the transformed sequences. In \cite{popivanov2002icde}, it is further observed that, other than orthonormal wavelets, bi-orthonormal ones can also be used for efficient similarity search over wavelet-indexed time series data, demonstrating several such wavelets that outperform the Haar wavelet in terms of precision and performance. In addition, an alternative approach to the $k$-nearest neighbor search over time series data is introduced in \cite{kashyap2011kdd}. The proposed method accesses the coefficients of Haar-wavelet-transformed time series through a sequential scan over step-wise increasing resolutions.

State-of-the-art approaches for time series indexing comprise methods based on the {\em Symbolic Aggregate Approximation} (SAX) representation \cite{jessica2007dmkd}. This is derived from the {\em Piecewise Aggregate Approximation} (PAA) representation of a time series \cite{keogh2001paa,faloutsos2000vldb}, by quantizing the segments of its PAA representation on the $y$-axis. The first attempt to leverage the potential of the SAX representation was presented in \cite{shieh2008kdd}, introducing the indexable Symbolic Aggregate Approximation ($i$SAX), capable of a multi-resolution representation for time series. The iSAX index was further extended to $i$SAX 2.0 in \cite{camerra2010icdm} by enabling bulk loading of time series data. Its next version is the $i$SAX2+ index \cite{camerra2014kais}, which handles better the expensive I/O operations caused by the aggressive node splitting while building the index. Finally, the ADS+ index \cite{zoumpatianos2014sigmod} is another extension of $i$SAX, which attempts to overcome the still significantly expensive index build time by adaptively building the index while processing the workload of queries issued by the user. A comprehensive overview and comparison of the time series indexing approaches based on the SAX representation is presented in \cite{palpanas2016bigsm}.

However, none of the above approaches supports geolocated time series, and thus cannot efficiently process hybrid queries combining time series similarity with spatial proximity.


\noindent \emph{Spatio-Textual Indices.} There is an increasing amount of spatio-textual objects, e.g., Points of Interest (PoI) with textual descriptions, geotagged tweets or posts in social media, etc. This has motivated research on hybrid spatial-keyword queries combining location-based predicates with keyword search. Main query types include the \emph{Boolean Range Query}, which retrieves all objects that contain a given set of keywords and are located within a specified spatial range; the \emph{Boolean $k$NN Query}, which returns the $k$ nearest objects to a specific location and contain the given keywords; and the \emph{Top-$k$ $k$NN Query}, which finds the top-$k$ objects according to an objective function that assigns hybrid scores to objects based on both their keyword similarity and spatial proximity to the query object \cite{chen2013pvldb}.

To evaluate such queries efficiently, the main idea is to construct hybrid index structures that simultaneously partition the data in both dimensions, spatial and textual. Essentially, this implies combining a spatial index structure (e.g., R-tree, Quadtree, Space-Filling Curve) with a textual index (e.g., inverted file, signature file). Depending on their form, the resulting variants can be characterized either as {\em spatial-first} or {\em textual-first} indices \cite{christoforaki2011cikm}. One of the most fundamental and characteristic ones is the IR-tree \cite{cong2009vldb,zhisheng2011tkde}, which extends the R-tree by augmenting the contents of each node with a pointer to an inverted file indexing terms and documents contained in its sub-tree. Several other hybrid spatio-textual indices extending the R-tree (or R$^*$-tree) have been proposed, such as the IR$^2$-tree \cite{defelipe2008icde}, the KR$^*$-Tree \cite{hariharan2007ssdbm}, SKI \cite{cary2010ssdbm} and S2I \cite{rocha2011ssd}, while methods based on space filling curves include SF2I \cite{chen2006sigmod} and SFC-QUAD \cite{christoforaki2011cikm}.

Our approach is based on the R-tree \cite{Guttman1984} for the spatial indexing part. Recall that an R-tree organizes a hierarchy of nested $d$-dimensional rectangles. Each node corresponds to a disk page and represents the MBR of its children or, for leaf nodes, the MBR of its contained geometries. The number of entries per node (excluding the root) is between a lower bound $m$ and a maximum capacity $M$. Query execution in R-trees starts from the root. MBRs in any visited node are tested for intersection against a search region. Qualifying entries are recursively visited until the leaf level or until no further overlaps are found. Several paths may be probed, because multiple sibling entries could overlap with the search region.


\noindent \emph{Correlated time series.} Identifying similar subsequences between time series also indicates some {\em correlation} between them. Several approaches compute pairwise statistics (e.g., Pearson correlation, beta values) especially in streaming time series \cite{zhu2002statstream,cole2005fast,papadimitriou2006local}. There are also works concerning {\em co-evolving} time series data, either towards detecting and correcting missing values \cite{yongjie2015fast} or mining typical patterns and points of variation to achieve a meaningful segmentation of large time series \cite{matsubara2014autoplait}. However, none of these approaches is applicable to our setting, where we require similarity in the time series values. 

\noindent \emph{Time series clustering.} Our work also relates to {\em clustering of time series}, where methods perform either partitioning or density-based clustering. In the former class, algorithms typically partition the time series into $k$ clusters. Similarly to iterative refinement employed in $k$-means, the $k$-Shape partitioning algorithm~\cite{Paparrizos:2015:KEA:2723372.2737793,Paparrizos:2017:FAT:3086510.3044711} aims to preserve the shapes of time series assigned to each cluster by considering the shape-based distance, a normalized version of the cross-correlation measure between time series. In contrast, density-based clustering methods are able to identify clusters of time series with arbitrary shapes. YADING~\cite{Ding:2015:YFC:2735479.2735481} is a highly efficient and accurate such algorithm, which consists of three steps: it first samples the input time series also employing PAA (Piecewise Aggregate Approximation) to reduce the dimensionality, then applies multi-density clustering over the samples, and finally assigns the rest of the input to the identified clusters. However, clustering methods consider time series in their entirety and not matching subsequences as we consider in this work.

\noindent \emph{Discovery of movement patterns in trajectories.} Our work also relates to approaches for discovering clusters of moving objects, in particular a type of movement patterns that is referred to as {\em flocks}~\cite{gudmundsson2006computing}. A flock is a group of at least $m$ objects moving together within a circular disk of diameter $\epsilon$ for at least $\delta$ consecutive timestamps. Finding an exact flock is NP-hard, hence this work suggests an \textit{approximate} solution to find the \textit{maximal} flock from a set of trajectories using computational geometry concepts. In \cite{benkert2008reporting}, another \textit{approximate} solution for detecting all flocks is based on a skip-quadtree that indexes sub-trajectories. Flock discovery over {\em streaming} positions from moving objects was addressed in \cite{vieira2009line}. This \textit{exact} solution discovers flock disks that cover a set of points at each timestamp. Their flock discovery algorithm finds candidate flocks per timestamp and joins them with the candidate ones from the previous timestamps, reporting a flock as a result when it exceeds the time constraint $\delta$. An improvement over this technique was presented in \cite{tanaka2015efficient}, using a \textit{plane sweeping} technique to accelerate detection of object candidates per flock at each timestamp, while an inverted index speeds up comparisons between candidate disks across time. In our setting, detection of bundles is similar to flocks, thus for our baseline method we adapt the algorithm from \cite{vieira2009line}.

All aforementioned approaches focus exclusively on combining spatial queries with keyword search. To the best of our knowledge, our work is the first one to address geolocated time series, combining spatial queries with similarity search for time series.}

}
%%%%%%%%%%%%%%%%%%%%%%%%%%%%%%%%%%%%%%%%%