Time series are important in many applications, ranging from the Web to finance and biology. Indexing and mining time series data has attracted a lot of interest, both from the database and data mining communities \cite{camerra2014kais,ding2008pvldb,shieh2008kdd}. Nevertheless, support for \emph{geolocated} time series is still lacking. Geolocated time series are time series that are produced at, or otherwise associated with, specific locations.

Geolocated time series are encountered in numerous applications. In geosocial networks, user check-ins associate locations with time series representing the number of visitors across time. In geomarketing or mobile advertisement, geolocated time series can be used to identify nearby places with similar visiting patterns. Geolocated time series are also generated by geotagged user posts, photos, micropayments, etc. Other examples involve time series generated by sensors installed at fixed locations, such as noise, weather or pollution sensors. In the \textit{DAIAD} project\footnote{\url{http://daiad.eu/}}, smart water meters are installed at households within a city to measure, analyze and mine water consumption patterns towards promoting more intelligent and efficient water use. Finding households in a region or close to a given location that exhibit consumption patterns similar to a given one can offer precious insights.

All such applications need to index geolocated time series to allow efficient similarity search based on both {\em spatial proximity} and {\em time series similarity}. Several approaches have been proposed for time series similarity, efficiently indexing large amounts of time series data. One well-studied family of approaches includes wavelet-based methods \cite{chan1999icde}, which rely on \emph{Discrete Wavelet Transform} \cite{graps1995cse} to reduce the dimensionality of time series and generate an index using the coefficients of the transformed sequences. Another line of work employs a \emph{Symbolic Aggregate Approximation} (SAX) representation of time series \cite{jessica2007dmkd}, introducing a series of indices, such as $i$SAX~\cite{shieh2008kdd}, $i$SAX 2.0~\cite{camerra2010icdm}, $i$SAX2+~\cite{camerra2014kais}, and ADS+~\cite{zoumpatianos2014sigmod}.

However, to the best of our knowledge, none of the existing works so far has considered the specific case of geolocated time series. All aforementioned indices aim at efficiently supporting similarity search for time series; in case that the analyzed time series are associated with a spatial attribute and issued queries involve spatial filters, these need to be treated independently. Thus, for queries employing both types of predicates, this implies evaluating each predicate separately. This can be done by first using  a time series index to retrieve similar time series and then applying the spatial predicate on the results, or vice versa, by employing a spatial index to evaluate the spatial predicate and then filter the results according to their similarity with the query time series.

In this paper, we propose a {\em hybrid index} for efficiently supporting similarity search on geolocated time series combining both spatial proximity and time series similarity. The proposed index, called \textit{\tsr}, is an extension of the R-tree spatial index. In the \tsr, each node is augmented with additional information corresponding to the bounds of the time series contained in its subtree, in addition to the standard \emph{Minimum Bounding Rectangle} (MBR) denoting the spatial bound of its contents. Maintaining both kinds of bounds in each node allows to prune the search space simultaneously in the spatial dimension and in the time series dimension while traversing the index. Thus, the number of required node accesses is significantly reduced, since we only retrieve the contents of nodes that may actually contain objects satisfying both types of predicates.

In addition, we propose an optimized variant, the \textit{\ctsr}, with its nodes having entries with more refined bounds by bundling together similar time series. This allows to compute and maintain tighter bounds for each individual bundle, hence increasing pruning effectiveness. To allow for a larger number of bundles in nodes at higher levels in the tree hierarchy, we exploit \emph{Piecewise Aggregate Approximation} \cite{keogh2001paa,faloutsos2000vldb} to trade off between the number of bundles and the resolution of the bounding time series for each bundle.

Our proposed approach follows a similar rationale to that applied in {\em hybrid spatio-textual indices} \cite{chen2013vldb}. In that line of research, several variants of hybrid indices (e.g., the IR-tree \cite{cong2009vldb}) have been proposed to tackle the problem of combining spatial predicates with keyword search. Constructing a hybrid index that combines spatial and textual pruning has been shown to speedup processing of hybrid spatial-keyword queries. Motivated by this, our goal is to provide similar improvements for queries on geolocated time series data. Nevertheless, spatio-textual indices are designed specifically for keyword search and typically rely on inverted indices for the textual part, hence they are not applicable to queries that involve similarity of time series where the sequence of values is important.

Summarizing, our main contributions are as follows:

\begin{itemize}
 \item We address the problem of similarity search for geolocated time series, via hybrid boolean or top-$k$ queries combining both spatial proximity and time series similarity.
 \item We propose the \tsr, a hybrid index for geolocated time series, extending the spatial R-tree and augmenting each node with appropriate time series bounds.
 \item We further optimize the \tsr to derive a more efficient variant, the \ctsr, that clusters the time series of each subtree to derive and maintain tighter bounds for pruning.
 \item We experimentally validate our proposed approach using real-world datasets from different application use cases, showing that our hybrid indices can effectively allow simultaneous pruning of the search space in both spatial and time series domains, significantly reducing the required number of node accesses.
\end{itemize}

The rest of the paper is organized as follows. Section \ref{sec:related} reviews related work. Section \ref{sec:problem} introduces the distance measures and hybrid query variants on geolocated time series data. Sections \ref{sec:tsrtree} and \ref{sec:ctsrtree} present the \tsr and \ctsr indices. Section \ref{sec:evaluation} reports our experimental results, and Section \ref{sec:conclusions} concludes the paper.