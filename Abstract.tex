% !TEX encoding = UTF-8 Unicode
% !TEX root = ThesisGchatzi.tex

\chapter{Abstract} \label{chap:abstract}

Time series are generated and stored at a vastly increasing rate in many industrial and research applications, including the Web and the Internet of Things, public utilities, finance, astronomy, biology, and many more. A significant portion concerns \textit{geolocated} time series, i.e., those generated at, or otherwise associated with specific locations. Although several works have focused on efficient time series \textit{similarity search}, there has been limited attention to the inherent challenge that geolocated time series introduce for hybrid queries on both spatial proximity and time series similarity. To efficiently process such queries, we propose a \textit{hybrid index}, called \btsr. Furthermore, we address the problem of \textit{hybrid similarity joins} over such geolocated time series. We propose both centralized and MapReduce-based algorithms for performing such join operations using spatial-only, time series-only, and hybrid indices.

Although several works exist for time series \textit{visualization} and visual analytics in general, there is a lack of efficient techniques for visual exploration and analysis of geolocated time series in particular. In this Thesis, we present two approaches that rely on hybrid indices to allow for efficient map-based visual exploration and summarization of geolocated time series data. In particular, we use the \btsr index and we introduce a new variant of the standard \isax index, called \hisax. We describe the structure of the new index and show how they can be directly exploited to produce map-based visualizations of geolocated time series at different levels of granularity. 

Apart from traditional similarity search, we also consider the problem of detecting \textit{locally} similar pairs and groups, called \textit{bundles}, over \textit{co-evolving} time series. These are pairs or groups of subsequences whose values do not differ by more than a predefined threshold for a number of consecutive timestamps, thus indicating common local patterns and trends. We propose a filter-verification technique that only examines candidate matches at judiciously chosen checkpoints across time. In the same line of work, we consider hybrid queries for retrieving geolocated time series based on filters that combine spatial distance and time series local similarity. To efficiently support such queries, we introduce the \sbtsr index, an extension of \btsr that further optimizes local similarity search.

Finally, we focus on large-scale \textit{forecasting} on big time series data. Specifically, we introduce FML-kNN, a novel distributed processing framework for Big Data that performs probabilistic classification and regression. The framework's core is consisted of a $k$-nearest neighbor joins algorithm which, contrary to similar approaches, is executed in a single distributed session and is able to operate on very large volumes of data of variable granularity and dimensionality.

Throughout this Thesis, we experimentally and empirically evaluate our work using synthetic and real-world datasets from diverse domains, against baseline and state-of-the-art existing methods, demonstrating the efficiency and superiority of our approaches.

\chapter{Περίληψη}

\selectlanguage{greek}

Στις μέρες μας, σε πολλές βιομηχανικές και ερευνητικές εφαρμογές (π.χ., διαδίκτυο των πραγμάτων, αστρονομία, οικονομικά, βιολογία) δημιουργείται και αποθηκεύεται μεγάλος όγκος δεδομένων \textit{χρονοσειρών}. Ένα σημαντικό ποσοστό αυτών αποτελούν οι \textit{γεωχωρικές} χρονοσειρές, δηλαδή εκείνες οι οποίες δημιουργούνται και σχετίζονται με συγκεκριμένες τοποθεσίες. Τα τελευταία χρόνια, πληθώρα επιστημονικών άρθρων μελετά μεθόδους \textit{αναζήτησης ομοιότητας} σε δεδομένα χρονοσειρών αψηφώντας τη γεωχωρική τους υπόσταση, η οποία θα επέτρεπε παρόμοια ερωτήματα βασισμένα --εκτός από την ομοιότητα στο πεδίο του χρόνου-- στη χωρική εγγύτητα των χρονοσειρών. Στη διατριβή αυτή, παρουσιάζουμε ένα \textit{υβριδικό} ευρετήριο με το όνομα \btsr, το οποίο μπορεί αποδοτικά να απαντήσει τέτοιου είδους υβριδικά ερωτήματα αναζήτησης ομοιότητας. Επιπροσθέτως, επικεντρωνόμαστε στο πρόβλημα των \textit{υβριδικών ενώσεων ομοιότητας} σε δεδομένα γεωχωρικών χρονοσειρών. Για την επίλυσή του, προτείνουμε κεντρικούς και κατανεμημένους αλγορίθμους βασισμένους στη μέθοδο MapReduce, με τη χρήση υβριδικών και μη ευρετηρίων.

Πληθώρα επιστημονικών άρθρων επικεντρώνεται στην \textit{οπτικοποίηση} και οπτική ανάλυση δεδομένων χρονοσειρών. Η αποδοτική οπτική εξερέυνηση γεωχωρικών χρονοσειρών, όμως, δεν έχει μελετηθεί επαρκώς. Στην παρούσα διατριβή, παρουσιάζουμε δυο προσεγγίσεις βασισμένες σε υβριδικά ευρετήρια, οι οποίες επιτρέπουν την αποδοτική εξερεύνηση δεδομένων γεωχωρικών χρονοσειρών μεγάλου όγκου με τη χρήση οπτικοποιήσεων σε χάρτη. Για την πρώτη, χρησιμοποιούμε το προαναφερθέν υβριδικό ευρετήριο \btsr. Η δεύτερη προσέγγιση βασίζεται σε μια επέκταση του υπάρχοντος ευρετηρίου χρονοσειρών \isax. Συγκεκριμένα, παρουσιάζουμε τη δομή του νέου ευρετηρίου και μεθόδους αποδοτικής οπτικοποίησης τέτοιων δεδομένων.

Εκτός των ανωτέρω, στην παρούσα διδακτορική διατριβή, επικεντρωνόμαστε στο πρόβλημα εντοπισμού ζευγαριών η ομάδων \textit{τοπικά} όμοιων συν-εξελισσόμενων χρονοσειρών. Συγκεκριμένα, τα ζευγάρια (ή ομάδες) αυτά αποτελόυνται από χρονοσειρές των οποίων οι τιμές σε οποιαδήποτε υποακολουθία τους δεν διαφέρουν περισσότερο από ένα δωθέν κατώφλι. Ο εντοπισμός τέτοιων ζευγαριών (ή ομάδων) μπορεί να φανερώσει χρήσιμα κοινά τοπικά μοτίβα και τάσεις σε δεδομένα χρονοσειρών. Για την εύρεσή τους, προτείνουμε μια μέθοδο φιλτραρίσματος-επαλήθευσης, η οποία επικεντρώνεται σε συγκεκριμένα σημεία ελέγχου στο πεδίο του χρόνου, επιταγχύνοντας τη διαδικασία. Παράλληλα, προτείνουμε μεθόδους απάντησης υβριδικών ερωτημάτων τοπικής ομοιότητας σε δεδομένα γεωχωρικών χρονοσειρών μεγάλου όγκου. Για την υποστήριξη τέτοιων ερωτημάτων, εισάγουμε μια επέκταση του ευρετηρίου \btsr, με το όνομα \sbtsr, το οποίο βελτιστοποιεί την βασισμένη σε τοπική ομοιότητα αναζήτηση.

Στο πλαίσιο της \textit{πρόβλεψης} δεδομένων χρονοσειρών μεγάλης κλίματας, παρουσιάζουμε ένα παράλληλο και κατανεμημένο πλαίσιο επεξεργασίας, το οποίο εκτελεί αποδοτικά κατηγοριοποίηση και παλινδρόμηση. Ο κεντρικός αλγόριθμός του πλαισίου είναι βασισμένος στη μέθοδο ενώσεων $k$-πλησιέστερων γειτόνων και μπορεί να εκτελεστεί σε μια παράλληλη συνεδρία --σε αντίθεση με παρόμοιες μεθόδους--, επιτρέποντας, έτσι, την εκτέλεση σε δεδομένα μεγάλου όγκου, διαφόρων βαθμών λεπτομέρειας και διαστατικότητας.

Τέλος, όλοι οι αλγόριθμοι και μέθοδοι που παρουσιαζονται στην διατριβή αυτή αξιολογούνται πειραματικά και εμπειρικά, με τη χρήση συνθετικών ή δεδομένων παργματικού κόσμου. Συγκεκριμένα, συγκρίνονται με βασικές, ή υπάρχουσες μεθόδους αιχμής (state-of-the-art), αποδεικνύοντας την υπεροχή τους και επιβεβαιώνοντας την αποδοτικότητά τους.